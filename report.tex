\documentclass[a4paper,12pt]{article}

\usepackage[left=1in, right=1in, top=0.5in, bottom=1in]{geometry}
\usepackage[latin2]{inputenc}
\usepackage{amsmath}
\usepackage{graphicx}
\usepackage[T1]{fontenc}
\usepackage{color}
\usepackage{amssymb}
%Some useful macros

\definecolor{Cgreen}{rgb}{0,0.6,0}
\definecolor{Cblue}{rgb}{0,0.39,0.61}

\newcounter{ohNoteCounter}
\newcommand{\ohnote}[1]{{\scriptsize  \color{Cgreen} $\clubsuit$~\refstepcounter{ohNoteCounter}\textsf{[OH]$_{\arabic{ohNoteCounter}}$:{#1}}}}

\newcounter{jpNoteCounter}
\newcommand{\jpnote}[1]{{\scriptsize  \color{Cblue} $\blacksquare$ \refstepcounter{jpNoteCounter}\textsf{[JP]$_{\arabic{jpNoteCounter}}$:{#1}}}}

\IfFileExists{.notes_disabled}{
	\renewcommand{\jpnote}[1]{}
	\renewcommand{\ohnote}[1]{}
}



\usepackage{times}

\begin{document}
  \pagestyle{empty}
  \ohnote{This is useful way to leave notes. Both of us have command tu leave notes. I have \textbackslash ohnote, you have \textbackslash jpnote. Notes in whole document can be disabled by creating file with name ``.notes\_disabled''. }

  \jpnote{Your notes will look like this }

  \section{Preliminaries}
  Let $f : \mathbf{x} \mapsto \mathcal{R}$ be a function that maps a point $\mathbf{u} = (u_j,u_i)$ to a real value $x_{ij}$, where $(u_j,u_i) \in [0,N_j] \otimes [0,N_i]$.  So $f$ gives the raw intensity value at each point in the image.  Denote $\mathbf{x_i}$ and$\mathbf{x_j}$  as the row and column of pixels at row $i$ and column $j$ respectively.  

  Define the Markov model as 

  \begin{eqnarray*}
    \mathcal{K} &=& \{ s_k | \text{\,column\,} k \text{\,of a letter, or column of whitespace \,} \} \\
    P(x_{ij}) &=& \text{probability of observing pixel intensity\,} x_{ij} \\
    P(s_k) &=& \text{probability of observing hidden state \,} s_{k} \\
    P(x_{ij}|s_k) &=& \text{probability of observing pixel intensity\,} x_{ij} \text{\,given\,} s_k  
  \end{eqnarray*}

  \section{Preprocessing}
  \subsection{Photometric Normalization}

  \section{Training}
  \subsection{Overview}
  We make the simplifying assumption that every two pixels intensities
  of a column are pairwise independent. Therefore the probability of
  an observed image column $\mathbf{x_j}$ given the hidden state $s_k$
  can be calculated as follows
  \begin{equation}
    p(\mathbf{x_j}|s_k) = \prod_{i=1}^{m} P(x_{ij}|s_k) ,
  \end{equation}
  
  Given an image of a license plate, a pixel in the image can be
  either part of the letter, which we call foreground, or part of the
  background of the license plate.  Thus the probability of observing
  image intenstity given hidden state $p(x_{ij}|s_k)$ is given by
  \begin{equation}
    P(x_{ij}|s_k)=P(x_{ij}|f_{js_k})P(f_{js_k})+P(x_{ij}|\bar{f_{js_k}})P(\bar{f_{js_k}})
  \end{equation}
  
  
  We assume that foreground and background pixels of the license plate
  are each normally distributed.  This motivates the choice of
  modeling the probability of observed pixel $x_{ij}$ given hidden
  state $s_k$ as the mixture of two gaussian distributions
  $\mathcal{N}_1(\sigma_1, \mu_1), \mathcal{N}_2(\sigma_2, \mu_2)$
  with mixing parameter so that
  \begin{equation}
    P(x_{ij} | s_k) = \alpha_{is_k} \mathcal{N}_1 + (1-\alpha_{is_k})
    \mathcal{N}_2 .
  \end{equation}

  \jpnote{need to talk about how the model enables different
    distributions for each row of each hidden state}

  
  The model is parameterized by Let $f : \mathcal{K} \mapsto
  \mathcal{T}, \mathcal{T} = \{ \mathbf{t_k} | \mathbf{t_k} =
  (\alpha_1 , \dots , \alpha_n)\}$ be a function that maps state $s_k$
  into the template $\mathbf{t_k}$ which is vector of numbers $\alpha
  \in [0, 1]$.

  The values of the unknown parameters $\mathcal{T}, \sigma_1,
  \sigma_2, \mu_1, \mu_2$ will be learned from the annotated license
  plate data set.

  

  \subsection{Expectation Maximization for Learning Emission
    probability distributions}
 obtain by running EM maximization
  algorithm on training set.
 

  

  \ohnote{Describe usage of EM alg.}


  \subsection{Setting Transition probabilities}
\end{document}

%--------------------------------------
% Here the document ends, code below is just as sample
%--------------------------------------


  You are writing a test and you are given a two choice question where
  the answer could be either A or B. You assume that the apriori probabilities $P(A) =
  P(B) = 0.5$ (i.e. there is equal a priori chance for each of the answers
  to be correct). After reading the question you estimate
  $P(A|\mbox{question}) = 0.8$. Further you know that the loss function
  has the following form:\newline

 \begin{tabular}{ccccc}
    $W(A, A)$ & = & $W(B, B)$ & = & $-2$,\\
    $W(A, B)$ & = & $W(B, A)$ & = & $4$,\\
    $W(A, ?)$ & = & $W(B, ?)$ & = & $0$,
  \end{tabular}\newline

  where $W(X, Y)$ is the loss function, $X$ is the correct answer, 
  $Y$ your decision, and ``?'' stands for not selecting any of the
  answers (i.e. you say that you do not know).

  \begin{enumerate}
    \item What will be your decision if you wanted to minimalise the
      Bayes risk and why?
    \item At which value of $P(A|\mbox{question})$ it starts to be
      advantageous to start to say ``I do not know''?
  \end{enumerate}

  \hfill {\it (2 points)}

  \vspace*{1em}

  \noindent {\bf Task 2} \newline

  Given $p(x|2) \propto N(0,1)$. Find (approximately using the plot below) the Neyman-Pearson decision strategy (normal state is $1$ and dangerous state is $2$) under the assumption that allowed probability of overlooked danger is 2\% and
  \begin{enumerate}
    \item $p(x|1) \propto N(1,1)$\hfill {\it (1 point)}
    \item $p(x|1) \propto N(1,\sigma)$; $\sigma \gg 1\hfill$ {\it (1 point)}
    \item $p(x|1) \propto N(0,\sigma)$\hfill {\it (1 point)}
  \end{enumerate}
  Explain your calculation. Is there a case when the optimal decision strategy decides for dangerous state on 3 or more disconnected intervals of $X$?


Tracking is formulated as an inference problem. Let $X_i$ be an object's internal state at grame $i$, a random variable, and $Y_i$ the measurements obained in the $i$-th frame. There are three main problems then:
\begin{itemize}
\item {\it Prediction.} Given $y_0, \ldots, y_{i-1}$ compute $P(X_i|Y_0=y_0, \ldots, Y_{i-1} = y_{i-1})$.
\item {\it Data Association.} Identify the measurements from $i$-th frame, which tell us something about $X_i$.
\item {\it Correction.} Given $y_i$ (the relevant ones) compute $P(X_i|Y_0=y_0, \ldots, Y_i = y_i)$.
\end{itemize}
Further, two independence assumptions are made:
\begin{itemize}
\item Only the immediate past matters, i.e. $$P(X_i|X_1, \ldots, X_{i-1}) = P(X_i|X_{i-1})\;.$$
\item Measurements depend only on the current state $$P(Y_i, Y_j, \ldots, Y_k|X_i) = P(Y_i|X_i) P(Y_j, \ldots, Y_k|X_i)\;.$$
\end{itemize}
Without these assumptions, tracking is very difficult. Thanks to these assumptions, tracking has the structure of inference on a hidden Markov model.

{\it Inference.} Done inductively. Lets assume we have $P(X_0)$. Then, correction step is easy
$$P(X_0|Y_0=y_0) = \frac{P(y_0|X_0)P(X_0)}{P(y_0)} \propto P(y_0|X_0)P(X_0)\;.$$
Lets assume we have $P(X_{i-1}|y_0,\ldots,y_{i-1})$. The prediction step is then (using the independence assumptions):
\begin{eqnarray*}
P(X_i|y_0,\ldots,y_{i-1}) &=& \int P(X_i,X_{i-1}|y_0,\ldots,y_{i-1})dX_{i-1}\\
&=& \int P(X_i|X_{i-1},y_0,\ldots,y_{i-1}) P(X_{i-1}|y_0,\ldots,y_{i-1})dX_{i-1}\\
&=& \int P(X_i|X_{i-1})P(X_{i-1}|y_0,\ldots,y_{i-1})dX_{i-1}\;.
\end{eqnarray*}
The correction step is then
\begin{eqnarray*}
P(X_i|y_0,\ldots,y_i) &=& \frac{P(X_i,y_0,\ldots,y_i)}{P(y_0,\ldots,y_i)}\\
&=& \frac{P(y_i|X_i,y_0,\ldots,y_{i-1}) P(X_i|y_0,\ldots,y_{i-1}) P(y_0,\ldots,y_{i-1})}{P(y_0,\ldots,y_i)}\\
&=& P(y_i|X_i) P(X_i|y_0,\ldots,y_{i-1})\frac{P(y_0,\ldots,y_{i-1})}{P(y_0,\ldots,y_i)}\\
&=& \frac{P(y_i|X_i) P(X_i|y_0,\ldots,y_{i-1})}{\int P(y_i|X_i) P(X_i|y_0,\ldots,y_{i-1}) dX_i}
\end{eqnarray*}
With this formulation, the {\bf key issue} is to find a representation of the densities that
\begin{itemize}
\item is sufficiently accurate,
\item allows quick and easy inference.
\end{itemize} 
{\bf Taxonomy} of different cases found in literature:
\begin{itemize}
\item Pdfs are linear, measurement model is linear and noise is Gaussian, then Kalman filter.
\item When the model is non-linear, then extended Kalman filter (not always reliable)
\item Multi-modal representation of $P(x_i|y_0,\ldots,y_i)$ -- Particle filter
\item Determining relevant $y_i$'s -- Data association
\end{itemize}  






